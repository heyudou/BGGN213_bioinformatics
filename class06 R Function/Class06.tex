\documentclass[]{article}
\usepackage{lmodern}
\usepackage{amssymb,amsmath}
\usepackage{ifxetex,ifluatex}
\usepackage{fixltx2e} % provides \textsubscript
\ifnum 0\ifxetex 1\fi\ifluatex 1\fi=0 % if pdftex
  \usepackage[T1]{fontenc}
  \usepackage[utf8]{inputenc}
\else % if luatex or xelatex
  \ifxetex
    \usepackage{mathspec}
  \else
    \usepackage{fontspec}
  \fi
  \defaultfontfeatures{Ligatures=TeX,Scale=MatchLowercase}
\fi
% use upquote if available, for straight quotes in verbatim environments
\IfFileExists{upquote.sty}{\usepackage{upquote}}{}
% use microtype if available
\IfFileExists{microtype.sty}{%
\usepackage{microtype}
\UseMicrotypeSet[protrusion]{basicmath} % disable protrusion for tt fonts
}{}
\usepackage[margin=1in]{geometry}
\usepackage{hyperref}
\hypersetup{unicode=true,
            pdftitle={Class06},
            pdfauthor={Phoebe He},
            pdfborder={0 0 0},
            breaklinks=true}
\urlstyle{same}  % don't use monospace font for urls
\usepackage{color}
\usepackage{fancyvrb}
\newcommand{\VerbBar}{|}
\newcommand{\VERB}{\Verb[commandchars=\\\{\}]}
\DefineVerbatimEnvironment{Highlighting}{Verbatim}{commandchars=\\\{\}}
% Add ',fontsize=\small' for more characters per line
\usepackage{framed}
\definecolor{shadecolor}{RGB}{248,248,248}
\newenvironment{Shaded}{\begin{snugshade}}{\end{snugshade}}
\newcommand{\KeywordTok}[1]{\textcolor[rgb]{0.13,0.29,0.53}{\textbf{#1}}}
\newcommand{\DataTypeTok}[1]{\textcolor[rgb]{0.13,0.29,0.53}{#1}}
\newcommand{\DecValTok}[1]{\textcolor[rgb]{0.00,0.00,0.81}{#1}}
\newcommand{\BaseNTok}[1]{\textcolor[rgb]{0.00,0.00,0.81}{#1}}
\newcommand{\FloatTok}[1]{\textcolor[rgb]{0.00,0.00,0.81}{#1}}
\newcommand{\ConstantTok}[1]{\textcolor[rgb]{0.00,0.00,0.00}{#1}}
\newcommand{\CharTok}[1]{\textcolor[rgb]{0.31,0.60,0.02}{#1}}
\newcommand{\SpecialCharTok}[1]{\textcolor[rgb]{0.00,0.00,0.00}{#1}}
\newcommand{\StringTok}[1]{\textcolor[rgb]{0.31,0.60,0.02}{#1}}
\newcommand{\VerbatimStringTok}[1]{\textcolor[rgb]{0.31,0.60,0.02}{#1}}
\newcommand{\SpecialStringTok}[1]{\textcolor[rgb]{0.31,0.60,0.02}{#1}}
\newcommand{\ImportTok}[1]{#1}
\newcommand{\CommentTok}[1]{\textcolor[rgb]{0.56,0.35,0.01}{\textit{#1}}}
\newcommand{\DocumentationTok}[1]{\textcolor[rgb]{0.56,0.35,0.01}{\textbf{\textit{#1}}}}
\newcommand{\AnnotationTok}[1]{\textcolor[rgb]{0.56,0.35,0.01}{\textbf{\textit{#1}}}}
\newcommand{\CommentVarTok}[1]{\textcolor[rgb]{0.56,0.35,0.01}{\textbf{\textit{#1}}}}
\newcommand{\OtherTok}[1]{\textcolor[rgb]{0.56,0.35,0.01}{#1}}
\newcommand{\FunctionTok}[1]{\textcolor[rgb]{0.00,0.00,0.00}{#1}}
\newcommand{\VariableTok}[1]{\textcolor[rgb]{0.00,0.00,0.00}{#1}}
\newcommand{\ControlFlowTok}[1]{\textcolor[rgb]{0.13,0.29,0.53}{\textbf{#1}}}
\newcommand{\OperatorTok}[1]{\textcolor[rgb]{0.81,0.36,0.00}{\textbf{#1}}}
\newcommand{\BuiltInTok}[1]{#1}
\newcommand{\ExtensionTok}[1]{#1}
\newcommand{\PreprocessorTok}[1]{\textcolor[rgb]{0.56,0.35,0.01}{\textit{#1}}}
\newcommand{\AttributeTok}[1]{\textcolor[rgb]{0.77,0.63,0.00}{#1}}
\newcommand{\RegionMarkerTok}[1]{#1}
\newcommand{\InformationTok}[1]{\textcolor[rgb]{0.56,0.35,0.01}{\textbf{\textit{#1}}}}
\newcommand{\WarningTok}[1]{\textcolor[rgb]{0.56,0.35,0.01}{\textbf{\textit{#1}}}}
\newcommand{\AlertTok}[1]{\textcolor[rgb]{0.94,0.16,0.16}{#1}}
\newcommand{\ErrorTok}[1]{\textcolor[rgb]{0.64,0.00,0.00}{\textbf{#1}}}
\newcommand{\NormalTok}[1]{#1}
\usepackage{graphicx,grffile}
\makeatletter
\def\maxwidth{\ifdim\Gin@nat@width>\linewidth\linewidth\else\Gin@nat@width\fi}
\def\maxheight{\ifdim\Gin@nat@height>\textheight\textheight\else\Gin@nat@height\fi}
\makeatother
% Scale images if necessary, so that they will not overflow the page
% margins by default, and it is still possible to overwrite the defaults
% using explicit options in \includegraphics[width, height, ...]{}
\setkeys{Gin}{width=\maxwidth,height=\maxheight,keepaspectratio}
\IfFileExists{parskip.sty}{%
\usepackage{parskip}
}{% else
\setlength{\parindent}{0pt}
\setlength{\parskip}{6pt plus 2pt minus 1pt}
}
\setlength{\emergencystretch}{3em}  % prevent overfull lines
\providecommand{\tightlist}{%
  \setlength{\itemsep}{0pt}\setlength{\parskip}{0pt}}
\setcounter{secnumdepth}{0}
% Redefines (sub)paragraphs to behave more like sections
\ifx\paragraph\undefined\else
\let\oldparagraph\paragraph
\renewcommand{\paragraph}[1]{\oldparagraph{#1}\mbox{}}
\fi
\ifx\subparagraph\undefined\else
\let\oldsubparagraph\subparagraph
\renewcommand{\subparagraph}[1]{\oldsubparagraph{#1}\mbox{}}
\fi

%%% Use protect on footnotes to avoid problems with footnotes in titles
\let\rmarkdownfootnote\footnote%
\def\footnote{\protect\rmarkdownfootnote}

%%% Change title format to be more compact
\usepackage{titling}

% Create subtitle command for use in maketitle
\newcommand{\subtitle}[1]{
  \posttitle{
    \begin{center}\large#1\end{center}
    }
}

\setlength{\droptitle}{-2em}

  \title{Class06}
    \pretitle{\vspace{\droptitle}\centering\huge}
  \posttitle{\par}
    \author{Phoebe He}
    \preauthor{\centering\large\emph}
  \postauthor{\par}
      \predate{\centering\large\emph}
  \postdate{\par}
    \date{1/25/2019}


\begin{document}
\maketitle

\subsection{File reading(again!)}\label{file-readingagain}

Here we try to use \textbf{read.table()} and friends to input some
example data in to R lets insert a code chunk.

\begin{Shaded}
\begin{Highlighting}[]
\NormalTok{data1 <-}\StringTok{ }\KeywordTok{read.csv}\NormalTok{(}\StringTok{"https://bioboot.github.io/bggn213_S18/class-material/test1.txt"}\NormalTok{)}
\NormalTok{data2 <-}\StringTok{ }\KeywordTok{read.table}\NormalTok{(}\StringTok{'test2.txt'}\NormalTok{,}\DataTypeTok{sep=}\StringTok{'$'}\NormalTok{,}\DataTypeTok{header =}\NormalTok{T)}
\NormalTok{data3 <-}\StringTok{ }\KeywordTok{read.table}\NormalTok{(}\StringTok{'test3.txt'}\NormalTok{)}
\end{Highlighting}
\end{Shaded}

Try assign a function

\begin{Shaded}
\begin{Highlighting}[]
\NormalTok{df <-}\StringTok{ }\KeywordTok{data.frame}\NormalTok{(}\DataTypeTok{a=}\DecValTok{1}\OperatorTok{:}\DecValTok{10}\NormalTok{, }\DataTypeTok{b=}\KeywordTok{seq}\NormalTok{(}\DecValTok{200}\NormalTok{,}\DecValTok{400}\NormalTok{,}\DataTypeTok{length=}\DecValTok{10}\NormalTok{),}\DataTypeTok{c=}\DecValTok{11}\OperatorTok{:}\DecValTok{20}\NormalTok{,}\DataTypeTok{d=}\OtherTok{NA}\NormalTok{) }
\NormalTok{df}
\end{Highlighting}
\end{Shaded}

\begin{verbatim}
##     a        b  c  d
## 1   1 200.0000 11 NA
## 2   2 222.2222 12 NA
## 3   3 244.4444 13 NA
## 4   4 266.6667 14 NA
## 5   5 288.8889 15 NA
## 6   6 311.1111 16 NA
## 7   7 333.3333 17 NA
## 8   8 355.5556 18 NA
## 9   9 377.7778 19 NA
## 10 10 400.0000 20 NA
\end{verbatim}

\begin{Shaded}
\begin{Highlighting}[]
\NormalTok{normalize <-}\StringTok{ }\ControlFlowTok{function}\NormalTok{(a) \{}
\NormalTok{  a <-}\StringTok{ }\NormalTok{(a }\OperatorTok{-}\StringTok{ }\KeywordTok{min}\NormalTok{(a)) }\OperatorTok{/}\StringTok{ }\NormalTok{(}\KeywordTok{max}\NormalTok{(a) }\OperatorTok{-}\StringTok{ }\KeywordTok{min}\NormalTok{(a))}
\NormalTok{\}}
\NormalTok{df}\OperatorTok{$}\NormalTok{a <-}\StringTok{ }\KeywordTok{normalize}\NormalTok{(df}\OperatorTok{$}\NormalTok{a)}
\NormalTok{df}\OperatorTok{$}\NormalTok{b <-}\StringTok{ }\KeywordTok{normalize}\NormalTok{(df}\OperatorTok{$}\NormalTok{b)}
\NormalTok{df}\OperatorTok{$}\NormalTok{c <-}\StringTok{ }\KeywordTok{normalize}\NormalTok{(df}\OperatorTok{$}\NormalTok{c)}
\NormalTok{df}
\end{Highlighting}
\end{Shaded}

\begin{verbatim}
##            a         b         c  d
## 1  0.0000000 0.0000000 0.0000000 NA
## 2  0.1111111 0.1111111 0.1111111 NA
## 3  0.2222222 0.2222222 0.2222222 NA
## 4  0.3333333 0.3333333 0.3333333 NA
## 5  0.4444444 0.4444444 0.4444444 NA
## 6  0.5555556 0.5555556 0.5555556 NA
## 7  0.6666667 0.6666667 0.6666667 NA
## 8  0.7777778 0.7777778 0.7777778 NA
## 9  0.8888889 0.8888889 0.8888889 NA
## 10 1.0000000 1.0000000 1.0000000 NA
\end{verbatim}

Next try to optimize the function

\begin{Shaded}
\begin{Highlighting}[]
\NormalTok{df <-}\StringTok{ }\KeywordTok{data.frame}\NormalTok{(}\DataTypeTok{a=}\DecValTok{1}\OperatorTok{:}\DecValTok{10}\NormalTok{, }\DataTypeTok{b=}\KeywordTok{seq}\NormalTok{(}\DecValTok{200}\NormalTok{,}\DecValTok{400}\NormalTok{,}\DataTypeTok{length=}\DecValTok{10}\NormalTok{),}\DataTypeTok{c=}\DecValTok{11}\OperatorTok{:}\DecValTok{20}\NormalTok{,}\DataTypeTok{d=}\OtherTok{NA}\NormalTok{) }
\NormalTok{df}
\end{Highlighting}
\end{Shaded}

\begin{verbatim}
##     a        b  c  d
## 1   1 200.0000 11 NA
## 2   2 222.2222 12 NA
## 3   3 244.4444 13 NA
## 4   4 266.6667 14 NA
## 5   5 288.8889 15 NA
## 6   6 311.1111 16 NA
## 7   7 333.3333 17 NA
## 8   8 355.5556 18 NA
## 9   9 377.7778 19 NA
## 10 10 400.0000 20 NA
\end{verbatim}

\begin{Shaded}
\begin{Highlighting}[]
\NormalTok{rescale <-}\StringTok{ }\ControlFlowTok{function}\NormalTok{(x) \{}
\NormalTok{ rng <-}\KeywordTok{range}\NormalTok{(x)}
\NormalTok{ (x }\OperatorTok{-}\StringTok{ }\NormalTok{rng[}\DecValTok{1}\NormalTok{]) }\OperatorTok{/}\StringTok{ }\NormalTok{(rng[}\DecValTok{2}\NormalTok{] }\OperatorTok{-}\StringTok{ }\NormalTok{rng[}\DecValTok{1}\NormalTok{])}
\NormalTok{\}}
\NormalTok{df}\OperatorTok{$}\NormalTok{a <-}\StringTok{ }\KeywordTok{normalize}\NormalTok{(df}\OperatorTok{$}\NormalTok{a)}
\NormalTok{df}\OperatorTok{$}\NormalTok{b <-}\StringTok{ }\KeywordTok{normalize}\NormalTok{(df}\OperatorTok{$}\NormalTok{b)}
\NormalTok{df}\OperatorTok{$}\NormalTok{c <-}\StringTok{ }\KeywordTok{normalize}\NormalTok{(df}\OperatorTok{$}\NormalTok{c)}
\NormalTok{df}
\end{Highlighting}
\end{Shaded}

\begin{verbatim}
##            a         b         c  d
## 1  0.0000000 0.0000000 0.0000000 NA
## 2  0.1111111 0.1111111 0.1111111 NA
## 3  0.2222222 0.2222222 0.2222222 NA
## 4  0.3333333 0.3333333 0.3333333 NA
## 5  0.4444444 0.4444444 0.4444444 NA
## 6  0.5555556 0.5555556 0.5555556 NA
## 7  0.6666667 0.6666667 0.6666667 NA
## 8  0.7777778 0.7777778 0.7777778 NA
## 9  0.8888889 0.8888889 0.8888889 NA
## 10 1.0000000 1.0000000 1.0000000 NA
\end{verbatim}

\begin{Shaded}
\begin{Highlighting}[]
\NormalTok{rescale2 <-}\StringTok{ }\ControlFlowTok{function}\NormalTok{(x) \{}
\NormalTok{ rng <-}\KeywordTok{range}\NormalTok{(x,}\DataTypeTok{na.rm=}\NormalTok{T)}
\NormalTok{ (x }\OperatorTok{-}\StringTok{ }\NormalTok{rng[}\DecValTok{1}\NormalTok{]) }\OperatorTok{/}\StringTok{ }\NormalTok{(rng[}\DecValTok{2}\NormalTok{] }\OperatorTok{-}\StringTok{ }\NormalTok{rng[}\DecValTok{1}\NormalTok{])}
\NormalTok{\}}
\KeywordTok{rescale2}\NormalTok{(}\KeywordTok{c}\NormalTok{(}\DecValTok{1}\NormalTok{,}\DecValTok{2}\NormalTok{,}\OtherTok{NA}\NormalTok{,}\DecValTok{3}\NormalTok{,}\DecValTok{10}\NormalTok{))}
\end{Highlighting}
\end{Shaded}

\begin{verbatim}
## [1] 0.0000000 0.1111111        NA 0.2222222 1.0000000
\end{verbatim}

Another way to optimize things

\begin{Shaded}
\begin{Highlighting}[]
\NormalTok{rescale3 <-}\StringTok{ }\ControlFlowTok{function}\NormalTok{(x, }\DataTypeTok{na.rm=}\OtherTok{TRUE}\NormalTok{, }\DataTypeTok{plot=}\OtherTok{FALSE}\NormalTok{) \{}
  \ControlFlowTok{if}\NormalTok{(na.rm) \{}
\NormalTok{    rng <-}\KeywordTok{range}\NormalTok{(x, }\DataTypeTok{na.rm=}\NormalTok{T)}
\NormalTok{  \} }\ControlFlowTok{else}\NormalTok{ \{}
\NormalTok{    rng <-}\KeywordTok{range}\NormalTok{(x)}
\NormalTok{  \}}
  \KeywordTok{print}\NormalTok{(}\StringTok{"Hello"}\NormalTok{)}
\NormalTok{  answer <-}\StringTok{ }\NormalTok{(x }\OperatorTok{-}\StringTok{ }\NormalTok{rng[}\DecValTok{1}\NormalTok{]) }\OperatorTok{/}\StringTok{ }\NormalTok{(rng[}\DecValTok{2}\NormalTok{] }\OperatorTok{-}\StringTok{ }\NormalTok{rng[}\DecValTok{1}\NormalTok{])}
  \CommentTok{#return(answer)}
  \KeywordTok{print}\NormalTok{(}\StringTok{"is it me you are looking for?"}\NormalTok{)}
  \ControlFlowTok{if}\NormalTok{(plot) \{}
    \KeywordTok{print}\NormalTok{(}\StringTok{"please don't ever sing again"}\NormalTok{)}
\NormalTok{  \}}
  \KeywordTok{plot}\NormalTok{(answer, }\DataTypeTok{typ=}\StringTok{"b"}\NormalTok{, }\DataTypeTok{lwd=}\DecValTok{4}\NormalTok{)}
  \KeywordTok{print}\NormalTok{(}\StringTok{"I can see it in ..."}\NormalTok{)}
  \KeywordTok{return}\NormalTok{(answer)}
\NormalTok{\}}
\KeywordTok{rescale3}\NormalTok{(}\DecValTok{1}\OperatorTok{:}\DecValTok{10}\NormalTok{,}\DataTypeTok{plot=}\OtherTok{TRUE}\NormalTok{)}
\end{Highlighting}
\end{Shaded}

\begin{verbatim}
## [1] "Hello"
## [1] "is it me you are looking for?"
## [1] "please don't ever sing again"
\end{verbatim}

\includegraphics{Class06_files/figure-latex/unnamed-chunk-5-1.pdf}

\begin{verbatim}
## [1] "I can see it in ..."
\end{verbatim}

\begin{verbatim}
##  [1] 0.0000000 0.1111111 0.2222222 0.3333333 0.4444444 0.5555556 0.6666667
##  [8] 0.7777778 0.8888889 1.0000000
\end{verbatim}

Section B

\begin{Shaded}
\begin{Highlighting}[]
\KeywordTok{library}\NormalTok{(bio3d)}
\NormalTok{s1 <-}\StringTok{ }\KeywordTok{read.pdb}\NormalTok{(}\StringTok{"4AKE"}\NormalTok{) }\CommentTok{# kinase with drug}
\end{Highlighting}
\end{Shaded}

\begin{verbatim}
##   Note: Accessing on-line PDB file
\end{verbatim}

\begin{Shaded}
\begin{Highlighting}[]
\NormalTok{s1.chainA <-}\StringTok{ }\KeywordTok{trim.pdb}\NormalTok{(s1, }\DataTypeTok{chain=}\StringTok{"A"}\NormalTok{, }\DataTypeTok{elety=}\StringTok{"CA"}\NormalTok{)}
\NormalTok{s1.b <-}\StringTok{ }\NormalTok{s1.chainA}\OperatorTok{$}\NormalTok{atom}\OperatorTok{$}\NormalTok{b}
\KeywordTok{plotb3}\NormalTok{(s1.b, }\DataTypeTok{sse=}\NormalTok{s1.chainA, }\DataTypeTok{typ=}\StringTok{"l"}\NormalTok{, }\DataTypeTok{ylab=}\StringTok{"Bfactor"}\NormalTok{)}
\end{Highlighting}
\end{Shaded}

\includegraphics{Class06_files/figure-latex/unnamed-chunk-6-1.pdf}

The number of atoms in this pdb 3459


\end{document}
